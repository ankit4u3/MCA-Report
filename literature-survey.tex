
\makeatletter
\def\@makechapterhead#1{%
  \vspace*{50\p@}%
  {\parindent \z@ \centering\normalfont
    \ifnum \c@secnumdepth >\m@ne
      \if@mainmatter
         \Large\bfseries \@chapapp\space \thechapter
        \par\nobreak
        \vskip 20\p@
      \fi
    \fi
    \interlinepenalty\@M
     \Large \bfseries #1\par\nobreak

    \vskip 40\p@
  }}
\def\@makeschapterhead#1{%
  \vspace*{50\p@}%
  {\parindent \z@ \centering
    \normalfont
    \interlinepenalty\@M
    \Large\bfseries  #1\par\nobreak
    \vskip 40\p@
  }}
\makeatother
\titlespacing*{\chapter}{0pt}{0pt}{12pt}

\section{Literature Review}

Treasure Hunts in libraries are more commonly the domain of primary and secondary 
schools, and are often created in a simple question/answer format. A review of the 
literature was therefore done on two different levels. Firstly, the Librarian was looking for 
interesting ideas or a ‘spin’ for the Hunt which would work well for mature students; and 
secondly, there was the need to investigate the pedagogical soundness of such an 
enterprise at the tertiary level. 
An initial exploration of the literature in September 2006 found little that created any 
inspirational spark. Treasure Hunts were often mentioned as part of orientation 
programmes, but never discussed in any great depth. By June 2007, it seemed as though 
everyone had had the same great idea simultaneously. The influx of pirate-themed 
Treasure Hunts, no doubt encouraged by the recent ‘Pirates’ movies, now inundate the 
internet. Several Treasure Hunts have been published and presented at conferences such 
as LIANZA 2006 (Telford, 2006) and MLA 2007 (Mongelia & Brown, 2007). Nowhere in the 
literature was there evidence of a Treasure Hunt quite like this one, where students 
collected clues, solved codes, and most importantly, visited many libraries and social 
facilities at once. Some had major internet components (Smith, 2007), where the 
‘treasure’ was ‘hidden’ in databases, and some were in reality offering nothing more than a 
guided tour (Langley, 2007) (Telford, 2006). Some Treasure Hunts were thoroughly 
uninspiring ("Welcome to the resources treasure hunt!," 2003), containing the vague and 
unstructured questions librarians often field at the beginning of a ‘reference interview’, and
requiring clues like “Go to chapter 2…”, which teaches nothing about the techniques of 
how and why a particular chapter is to be found. A number of reports on Treasure Hunts 
were negative, but usually in the context of Library-based Information Literacy Evaluation 
programmes. They strongly recommended Treasure Hunts should not be used to evaluate 
students’ information literacy ("Assignments to Test for Information Literacy Skills," 2006; 
"Creating Effective Library Assignments: A Guide for Faculty " 2005). Fortunately for our 
students, the Treasure Hunt was developed as a method to help them learn, gain 
confidence and explore, not to assess them. 
Investigation of the pedagogical soundness of this enterprise was much more 
encouraging. At the core is the well-researched concept of learner-centred, active learning 
techniques, where learning is embedded in authentic tasks, is collaborative, integrated, 
reflective, and socially constructed (Biggs, 2003; Ramsden, 2003). Accordingly, the 
activities reflected the multidisciplinary nature of the BOH programme by involving many 
stops to gather clues from a variety of locations, social and academic, where the students 
worked together to solve the clues, and gained some new friends in the process. The 
Treasure Hunt was also pitched at an appropriate level of difficulty, with opportunities for 
feedback and discussion. Many of the students in the course are teenagers, so it was 
important to acknowledge their predilections as ‘millennial’ students, and what that would 
mean in regard to their learning. The seven core traits (special, sheltered, confident, team 
oriented, conventional, pressured, and achieving) as coined by Howe and Strauss (2003), 
marry comfortably into the goals of learner-centred learning, and therefore the Treasure 
Hunt itself (Mongelia & Brown, 2007)

\subsection{Planning and Design}

During late August 2006, the idea of a Treasure Hunt was first tabled as a learner-centred 
activity that met the objectives of the induction programme and the Librarian’s 
expectations of acquiring base-level information literacy skills. Following a number of brain 
storming sessions, the Librarian used her recently acquired Project Management skills to 
create a detailed timeline, and a risk assessment matrix to identify risks early, and 
minimise them. For example, a common design feature of a Treasure Hunt is to locate a 
particular citation in a book and use it to find another resource. This often leads to the 
citation being underscored by participants, or even worse, the pertinent page being torn 
from the book by a particularly competitive student. Consequently, this Treasure Hunt only 
once required students to look inside a text. The rest of the questions could be answered 
from call numbers, the outside covers, or simply range guide signs. Other risks identified 
generally had a high impact, but a low likelihood, e.g. rain, power-cut no internet access, 
other libraries and other ‘Campus Hot Spots’ not being prepared to participate. Anticipated 
issues such as having hoards of students racing through their buildings simply required 
good organisation and preparation. The Librarian contacted all ‘Hot Spot’ service providers 
and Libraries well in advance, ensuring them of their minimal obligations (for example, 
handing out Clue Cards), and chose locations that could in fact cope with large numbers of 
students. All participating service providers immediately saw the benefit for themselves 
and supported the endeavour whole-heartedly. The main risk, which was the most timeconsuming to eliminate, was creating clues for each library that had logical connectivity and were error-free.All identified risks helped to shape the timeline. First was confirmation of topics to be covered at each library, forming the scaffold for the Hunt. The Librarian and Academic 
staff member worked together on this, ensuring both sides’ objectives were met. For 
instance, the students needed to know that research and study for the BOH would require 
a wide range of reading. They might go to the Central (humanities) Library for sociology 
texts, to the Science Library for nutrition or chemistry, to the Medical Library for drug 
handbooks, or to the Dental Library for discipline specific information. The Treasure Hunt 
demonstrated this multidisciplinary nature of their degree by choosing clues that interconnected. To illustrate: students had to find a particular journal article containing a specific reference from the Dental Library, and then had to find the cited journal in the 
Central Library. 

The Hunt went through several testing stages, with Library colleagues being used as 
testers on the last draft. Upon completion of this planning process the Treasure Hunt 
packs for each pair of students had to be made up and all staff briefed. We were ready to 
go! 

\section{The Treasure Hunt}

The event took place on the afternoon of 26th February 2006 during orientation week. 36 
students congregated in a teaching laboratory, where they were introduced to the 
Librarians, watched a short presentation on the art of library-based research skills, and 
participated in some basic hands-on practice in using the library catalogue. Then they 
received their Treasure Hunt packs which included instructions, maps of the campus and 
of all the libraries, a guide to the catalogue, a handout of the key points made in the 
presentation, and their first of five clue-cards. To ensure that some of the smaller Libraries 
and ‘Campus Hot Spots’ were not overwhelmed, the group was divided into pairs, and 
given different starting points. Ideally, no more than 4 pairs of students were ever at the 
same place at any one time. They were given 1.5 hours to complete the Hunt, and 
instructed to meet back at ‘base’ at 4pm to Crack the Code and win Treasure! 
The clue-card directed the student to a specific library, where they had to collect answers 
to 4 or 5 clues. From these specific letters would be used in order to complete the Code 
Cracker at the end of the Hunt. Often the answer to one clue was required to solve the 
next one. So for instance, in the Central Library, ‘What floor would you find Social 
Sciences books on?’ had an answer: ‘[Floor] ONE’. The next question read ‘Your map 
shows a Help Desk is on this floor, what does the blue sign above the desk actually say?’ 
The correct answer was “[Reserve and] GENERAL” The 6th letter of the word was the 
answer to Clue 16. Once the student had filled in all the clues on the card, they were 
directed to a staffed desk (usually a Reference Desk) where they could collect the next 
clue-card. Staff at the desks were only required to scan the completed card to ensure it 
was indeed complete, and to hand over the next clue. They were also provided with an 
answer sheet to assist if any students could not progress. Some staff did actively help the 
students. 
The five places students were required to visit were the four libraries: Dental, Medical, 
Science and Central, and one hotchpotch of places called, for the purposes of the 
Treasure Hunt, ‘Campus Hot Spots’. This included Student Health, Disability Information 
and Support, The Print Shop, and Clubs and Societies (owned by the Student 
Association). These ‘Hot Spots’ are geographically situated near each other, and clues for
these places were all on the one clue-card. Clubs and Societies was chosen as the next 
pick up point as they were accustomed to crowds, and had a roomy reception desk. 
As is the nature of the ‘traditional’ tour, the Treasure Hunt had to fulfil the basic 
requirement of ensuring the students simply knew what was where, and how to find things 
themselves next time. Therefore many clues had a location focus, like ‘Find the journal 
shelved directly after Diabetes Care. What is it?’ and this was extended by asking: ‘What is 
the main language used?’ These questions demonstrated that journals in the Medical 
Library are shelved alphabetically, and that a foreign sounding journal [Diabetologia] is not 
necessarily in a foreign language [it is written mainly in English]. The Hunt also showed 
that all libraries offer very similar services, but might name them differently, such as 
Information Desk versus Help Desk. Or, that although all Libraries use call numbers to 
shelve books, the call numbers for the same book could be slightly different from one 
library to the next. 
During the Hunt students were to use the catalogue three times in three different libraries, 
and in three different ways. This demonstrated that all the Library’s resources are listed 
on the same catalogue, and that information could be accessed in different ways, 
depending on the need. Upon their return at 4 o’clock, the pairs of students worked with others to complete the 
final task and submit the agreed answer to the Code Cracker. As individuals they all 
completed a single-page Quick-fire Quiz, designed to recap and reinforce their learning 
experiences. This time together was also an opportunity for students to summarise their 
learning and make a record for their own future reference. All students were asked to 
reflect on the activity and complete a brief evaluation with the specified intention that their 
responses would lead to changes for the Hunt at the beginning of 2008. The ‘treasure’ of 
the Treasure Hunt was not just successful completion of the Hunt, tangible rewards from 
an oral health products company were taken away. 
Two days later, the Librarians met the BOH class again for two reasons. One was to give 
out special prizes: three large “Dr. Rabbit” soft toys were given to three students who had 
correctly answered the final Quick-Fire Quiz (all correct entries were put in a hat, and the 
three names were randomly selected). The second purpose of our visit was to report back 
on their collective reflective feedback, and talk about improvements for the students next 
year. One month later, the Librarians met the BOH class again once the group had 
submitted assignments for Sociology and BOH, we asked them to reflect with the benefit 
of hindsight, the usefulness of the Treasure Hunt. 

\subsection{Discussion}

The students’ first learning experiences in developing library skills, and knowledge of the 
range of information available has had positive outcomes in their written work where we 
begin to see a wider variety of resources accessed than has been the case with previous 
first year students. 
New students were not aware of the wide range of services offered by the University and 
had appreciated this opportunity to visit the sites selected. Other units such as Student 
Job Search, and Unipol Sports Centre were suggested as useful ‘Hot Spots’ to explore 
next time. 
These students got to know some members of the class well in the time spent ‘hunting’ 
and they were beginning to establish a sense of collegiality. However, it is clear that we 
were quite naïve to believe this would not be a highly competitive exercise. 
Encouraging these students to explore the wider campus took them away from the Faculty 
of Dentistry and meant that they became more familiar with that larger environment. 
However, the larger environment at Otago is quite spread out, so the student feedback of 
needing more time next year will be heeded, as will our advice to wear sensible shoes! 
Remarks made by library staff alerted us to a few minor issues: time pressure meant that 
queuing ‘etiquette’ was largely ignored. Instructions were not always read as literally as 
required e.g. students looked for a ‘Service Desk’ at the Central Library, when the 
instructions told them to look for the ‘Help Desk’. 
Some mild confusion was created with the language used e.g. students were asked to find 
the ‘Social Sciences’ section (based on signage in the Central Library), but staff said that 
all resources at Central could be considered social sciences. 
We will do it again! The intention from the outset was to place the students at the centre of 
their learning experience, to provide them with a series of tasks that would be meaningful 
to them especially as a group with the characteristics previous experience had led us to anticipate, and to leave them in control of the process – there was no compulsion to 
complete any or all of the exercise. The students all strove to achieve complete answer 
sheets, there was a sense that not doing so would characterise ‘failure’ – even to the 
extent that one pair refused to return until they were satisfied with their work. There was a 
clear message from students that they were in control (once the initial 
instruction/background had been provided and the task set). Perhaps one or two felt this 
was a little ‘infra dig’ but the wholehearted involvement of the whole class made it a very 
worthwhile experience.
\subsection{Conclusion}
A carefully designed Treasure Hunt can be successfully conducted in a reasonably limited 
period of time and provide students with a stimulating and worthwhile learning experience