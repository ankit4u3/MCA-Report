



\usepackage{titlesec}
\titleformat{\chapter}[display]
{\normalfont\Large\bfseries}{\centering\chaptertitlename\ \thechapter}{12pt}{\Large}
\titlespacing*{\chapter}{0pt}{0pt}{10pt}


\chapter{ Introduction}
Treasure Hunt is a real-world, outdoor treasure hunting game using GPS-enabled devices. Participants navigate to a specific set of GPS coordinates and then attempt to find the geocache (container) hidden at that location.At its simplest level, geocaching requires these 8 steps:
\begin{itemize}



  \item Register with the game server.
  \item Visit the "Hide & Seek a Cache" page.
  \item Enter your name  and click "search."
  \item Choose any Game Locations from the list and click on its name.
  \item Click on Start Hunt from your Android  GPS Device.
  \item Use yourAndroid  device to assist you in finding the hidden Treasure Location.
   \item Share your Treasure Hunt stories and photos online.
 
\end{itemize}




\section{Locations & Rules }
Geocaches can be found all over the world. It is common for geocachers to hide caches in locations that are important to them, reflecting a special interest or skill of the cache owner. These locations can be quite diverse. They may be at your local park, at the end of a long hike, underwater or on the side of a city street.In its simplest form, a cache always contains a logbook or logsheet for you to log your find. Larger caches may contain a logbook and any number of items. These items turn the adventure into a true treasure hunt. You never know what the cache owner or visitors to the cache may have left for you to enjoy. Remember, if you take something, leave something of equal or greater value in return. It is recommended that items in a cache be individually packaged in a clear, zipped plastic bag to protect them from the elements.

People of all ages hide and seek geocaches, so think carefully before placing an item into a cache. Explosives, ammunition, knives, drugs and alcohol should not be placed in a cache. Respect local laws at all times.Please do not put food or heavily scented items in a cache. Animals have better noses than humans, and in some cases caches have been chewed through and destroyed because of food items in a cache.




\section{Trackables}
A Trackable is a sort of physical geocaching "game piece." You will often find them in geocaches or see them at geocaching gatherings. Each Trackable is etched with a unique code that can be used to log its movements on Geocaching.com as it travels in the real world. Some of these items have traveled hundreds of thousands of miles thanks to geocachers who move them from cache to cache!

There are three main types of Trackables: Travel Bug® Trackables, Geocoins and other Trackables.

A Travel Bug is a trackable tag attached to an item that geocachers call a "hitchhiker." Each Travel Bug has a goal set by its owner. Goals are typically travel-related, such as to visit every country in Europe or travel from coast to coast. Travel Bug Trackables move from cache to cache with the help of geocachers like you. See the "What do I do when I find a Trackable?" section of the guide for information on how you can help Trackables move.

Geocoins are customizable coins created by individuals or groups of geocachers as a kind of signature item or calling card. They function exactly like Travel Bug Trackables and should be moved to another cache, unless otherwise specified by their owners.

Other Trackable items come in various forms including patches, key rings and more. A common feature of Trackable items is that they bear a unique ID code and text noting that they are trackable at Treasure Hunt or Ankit APPS Google Play Store.




\begin{figure} [ht]
\left
\includegraphics[scale=0.5]{cache}\\
\caption{Treasure Hunt }
\label{the-label-for-cross-referencing}
\end{figure}













